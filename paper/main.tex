\documentclass[11pt,a4paper]{article}

% ----------------------------
% Core packages
% ----------------------------
\usepackage[utf8]{inputenc}
\usepackage[T1]{fontenc}
\usepackage{geometry}
\geometry{margin=1in}
\usepackage{graphicx}
\usepackage{amsmath, amssymb}
\usepackage{booktabs}
\usepackage{hyperref}
\usepackage{natbib}
\usepackage{float}
\usepackage{caption}
\usepackage{subcaption}
\usepackage[final]{microtype}
\microtypesetup{expansion=false}


% ----------------------------
% Algorithm box (FIXES your \Require error)
% Use algorithmicx + algpseudocode (NOT the old algorithmic)
% ----------------------------
\usepackage{algorithm}
\usepackage{algpseudocode}

% ----------------------------
% Code listings (short, reviewer-friendly)
% ----------------------------
\usepackage{listings}
\usepackage{xcolor}
\lstset{
  basicstyle=\ttfamily\small,
  breaklines=true,
  frame=single,
  numbers=left,
  numberstyle=\tiny,
  xleftmargin=2em,
  framexleftmargin=1.5em
}

% ----------------------------
% Title / metadata
% ----------------------------
\title{\textbf{Quant-Safe Explainable Artificial Intelligence for Dynamic Portfolio Management}\\
\vspace{0.15cm}\large{A Reproducible, Leakage-Resistant ML + SHAP Framework with Execution-Grade Accounting Logs}}

\author{
\textbf{Panagiotis Karmiris}\\
Independent Researcher\\
Email: \texttt{unbinder@msn.com}\\
ORCID: \texttt{(optional)}\\
\vspace{0.2cm}
}

\date{Version: \today}

\begin{document}
\maketitle

\begin{abstract}
Machine learning (ML) backtests in finance frequently overstate performance due to data leakage, non-point-in-time features, and evaluation procedures that inadvertently incorporate future information. This paper proposes a leakage-resistant, reproducible, and deployment-oriented framework---the \emph{Quant-Safe} architecture---combining (i) point-in-time feature engineering with explicit reporting lags, (ii) walk-forward evaluation with out-of-sample (OOS) explainability, and (iii) a robust portfolio translation layer with transaction-cost modeling and execution-grade accounting logs. We validate the framework on the Dow Jones Industrial Average (DJI) constituent universe over 2015--2025, using gradient-boosted trees and Shapley Additive Explanations (SHAP) to demonstrate that macro regime variables (e.g., interest-rate proxies) become dominant drivers during stress periods. The primary contribution is an engineering methodology enabling other researchers to reproduce, extend, and audit financial ML results with explicit controls against common failure modes.
\end{abstract}

\noindent\textbf{Keywords:} financial machine learning; data leakage; walk-forward validation; explainable AI; SHAP; portfolio construction; backtest overfitting; execution logging

\vspace{0.3cm}

% ==========================================================
\section{Introduction}
The empirical finance literature increasingly applies ML models to return prediction and cross-sectional asset selection \citep{gu2019empirical, feng2020deep}. However, many reported gains fail to survive realistic deployment conditions due to data leakage, improper temporal splits, and missing trading frictions \citep{bailey2017pbo, lopez2018advances}. In parallel, explainable AI (XAI) techniques such as SHAP offer a path to interpretability and regime diagnostics, reducing ``black-box'' risk in investment decision-making \citep{lundberg2017unified, molnar2020interpretable, babaei2025explainable}.

This study adapts XAI-driven investment modeling to a stricter engineering standard. Rather than emphasizing headline returns, we emphasize \emph{auditability}: point-in-time data semantics, leakage prevention, OOS explainability, and execution-consistent performance logging.

\subsection{Contributions}
We contribute:
\begin{enumerate}
  \item A \textbf{Quant-Safe architecture} defining operational safeguards for financial ML pipelines: reporting lags, point-in-time features, walk-forward evaluation, and OOS-only explainability.
  \item A \textbf{formalized Algorithm 1} and a \textbf{Data Leakage Failure Modes} checklist aimed at reviewer scrutiny and reproducible research.
  \item A \textbf{na\"ive vs Quant-Safe comparison table} illustrating why common evaluation shortcuts inflate results.
  \item A reproducible implementation with \textbf{execution-grade accounting logs} (daily mark-to-market equity, realized/unrealized P\&L by ticker, slippage proxies) and a Monte Carlo appendix for robustness.
\end{enumerate}

% ==========================================================
\section{Related Work}
ML asset pricing and return forecasting have advanced rapidly \citep{gu2019empirical, feng2020deep}. Yet systematic over-optimism in backtests is well documented, including backtest overfitting \citep{bailey2017pbo} and leakage introduced by non-point-in-time features and improper cross-validation \citep{lopez2018advances}. Interpretability frameworks such as SHAP \citep{lundberg2017unified} and general XAI guidance \citep{molnar2020interpretable, babaei2025explainable} motivate the use of explanation not merely as a diagnostic, but as a safeguard against spurious relationships.

On the portfolio side, we adopt robust, low-parameter constructions (top-$N$, inverse-volatility weights) consistent with the preference for simple, stable allocation rules under estimation error \citep{markowitz1952portfolio, clarke2013risk}. We also explicitly model transaction costs, which are central to realistic strategy evaluation \citep{almgren2001optimal, kissell2013science}.

% ==========================================================
\section{Data and Experimental Design}
\subsection{Universe and Horizon}
The research universe is the DJI constituent set used in the project codebase. The modeling target is a forward return proxy (as implemented in the pipeline), while the portfolio is rebalanced at a fixed schedule (monthly in the primary configuration).

\textbf{Limitation (Survivorship Bias):} If the study uses a modern constituent list retroactively, survivorship bias may inflate performance. The primary claim of this paper is the \emph{Quant-Safe methodology} (leakage prevention + OOS explainability), not the absolute historical return magnitude. Future work should incorporate point-in-time constituent membership or investable proxies \citep{lopez2018advances}.

\subsection{Macro and Risk Controls}
We incorporate a minimal macro block designed to be easy to reproduce:
\begin{itemize}
  \item Interest-rate proxy (e.g., 10Y yield series or yield-change z-scores)
  \item Risk/volatility proxy (e.g., VIX or realized volatility)
\end{itemize}
These variables enable regime attribution in SHAP and motivate risk scaling when volatility spikes.

% ==========================================================
\section{Quant-Safe Methodology}
\subsection{Design Principles}
The Quant-Safe architecture enforces:
\begin{itemize}
  \item \textbf{Point-in-time semantics:} every feature must be available at prediction time.
  \item \textbf{Reporting lags:} fundamentals are shifted by a disclosure lag to avoid ``trading on future filings''.
  \item \textbf{Temporal evaluation:} walk-forward or strictly OOS evaluation for research claims.
  \item \textbf{Explainability discipline:} SHAP computed on OOS folds only, preventing explanation leakage.
  \item \textbf{Execution consistency:} transaction costs, slippage proxies, and accounting-quality logs.
\end{itemize}

\subsection{Algorithm 1: Quant-Safe Pipeline}
\begin{algorithm}[H]
\caption{Quant-Safe Pipeline (Walk-Forward + OOS SHAP + Portfolio Layer)}
\label{alg:quantsafe}
\begin{algorithmic}[1]
\Require Asset prices $\{P_{i,t}\}$, macro variables $\{M_t\}$, optional fundamentals $\{F_{i,q}\}$ with reporting lag $\Delta$
\Require Prediction horizon $H$, rebalance schedule $\mathcal{R}$, cost model $\mathcal{C}$, top-$N$ selection rule
\State Align prices to trading calendar; compute returns and technical factors
\State Lag fundamentals: $\tilde{F}_{i,t} \leftarrow F_{i,q(t-\Delta)}$
\State Construct feature vector $X_{i,t} \leftarrow [\text{tech}_{i,t}, \tilde{F}_{i,t}, M_t]$
\State Label history where forward outcomes are observable: $y_{i,t} \leftarrow \text{Return}(t \rightarrow t+H)$
\For{$k = 1$ to $K$ walk-forward steps}
  \State Define train window $\mathcal{T}_k$ and test window $\mathcal{S}_k$ with strict time ordering
  \State Fit model $f_k$ on $\{(X_{i,t}, y_{i,t}) : t \in \mathcal{T}_k\}$
  \State Score OOS: $\hat{y}_{i,t} = f_k(X_{i,t})$ for $t \in \mathcal{S}_k$
  \State Compute SHAP on OOS only: $\phi_{i,t} \leftarrow \text{SHAP}(f_k, X_{i,t})$ for $t \in \mathcal{S}_k$
  \State Form portfolio at each rebalance $t \in \mathcal{R} \cap \mathcal{S}_k$:
    \Statex \quad Select top-$N$ by $\hat{y}_{i,t}$; weights via inverse volatility; apply caps/turnover constraints
  \State Simulate execution with $\mathcal{C}$; record trades and mark-to-market equity
\EndFor
\State Aggregate OOS predictions, performance metrics, and SHAP summaries across steps
\end{algorithmic}
\end{algorithm}

% ==========================================================
\subsection{Data Leakage Failure Modes (Reviewer Checklist)}
Reviewers frequently reject financial ML work due to unaddressed leakage. Table \ref{tab:leakage} lists common failure modes and explicit mitigations.

\begin{table}[H]
\centering
\caption{Common Data Leakage Failure Modes and Quant-Safe Mitigations}
\label{tab:leakage}
\begin{tabular}{p{3.6cm} p{5.6cm} p{5.6cm}}
\toprule
\textbf{Failure Mode} & \textbf{How It Appears} & \textbf{Quant-Safe Mitigation} \\
\midrule
Look-ahead via random split & High CV scores; fails live & Walk-forward / time-ordered OOS only \\
Non-point-in-time fundamentals & ``Predicts earnings surprises'' unrealistically & Explicit reporting lag $\Delta$ \\
Same-day macro availability & Using revised/late macro prints & Use release-aware series; conservative lag \\
Target leakage in features & Feature computed with $t{+}H$ data & Audit feature timestamps; unit tests \\
Survivorship bias in universe & Overstates long-run returns & Declare limitation; prefer point-in-time membership \\
Cost-free backtest & Unrealistically high turnover alpha & Cost model + slippage proxies \\
Explanation leakage & SHAP on train or post-fit full data & Compute SHAP only on OOS folds \\
\bottomrule
\end{tabular}
\end{table}

% ==========================================================
\subsection{Na\"ive ML Backtest vs Quant-Safe Evaluation}
Table \ref{tab:naive_vs_safe} clarifies why many ``state-of-the-art'' results do not survive deployment.

\begin{table}[H]
\centering
\caption{Comparison: Na\"ive ML Backtests vs Quant-Safe Architecture}
\label{tab:naive_vs_safe}
\begin{tabular}{p{4.0cm} p{5.5cm} p{5.3cm}}
\toprule
\textbf{Component} & \textbf{Na\"ive Implementation} & \textbf{Quant-Safe Implementation} \\
\midrule
Validation split & Random K-fold & Walk-forward (expanding/rolling) \\
Fundamentals & Use as-of values without lag & Lag by disclosure delay \\
Feature scaling & Global z-score & Cross-sectional or time-safe scaling \\
Explainability & SHAP on full dataset & SHAP strictly OOS per fold \\
Portfolio layer & Optimized weights, unconstrained & Top-$N$, inverse-vol, caps, turnover control \\
Trading frictions & Often omitted & Explicit costs + slippage proxies \\
Accounting & P\&L only & MTM equity + realized/unrealized by ticker + reconciliation \\
\bottomrule
\end{tabular}
\end{table}

% ==========================================================
\section{Modeling and Explainability}
\subsection{Model Choice}
We employ gradient-boosted decision trees due to strong performance in structured financial feature sets and robust handling of non-linear interactions \citep{chen2016xgboost, gu2019empirical}. Hyperparameters are tuned within the training window only, consistent with temporal safety \citep{lopez2018advances}.

\subsection{Out-of-Sample SHAP}
SHAP decomposes predictions into additive feature attributions under a cooperative game-theoretic framework \citep{lundberg2017unified}. In this study, SHAP values are computed \emph{only} on OOS test folds to preserve interpretability under real-time constraints and avoid explanatory leakage.

% ==========================================================
\section{Portfolio Construction, Costs, and Accounting}
\subsection{Portfolio Translation Layer}
We map predicted returns $\hat{y}_{i,t}$ into an actionable portfolio:
\begin{itemize}
  \item \textbf{Selection:} top-$N$ assets by $\hat{y}_{i,t}$.
  \item \textbf{Weights:} inverse-volatility weights to reduce risk concentration \citep{clarke2013risk}.
  \item \textbf{Rebalance:} monthly (primary configuration).
  \item \textbf{Turnover control:} trade only when ranks change materially.
  \item \textbf{Risk caps:} max weight per name; optional regime-based gross reduction.
\end{itemize}

\subsection{Trading Frictions}
We include a transaction cost and slippage proxy consistent with execution literature \citep{almgren2001optimal, kissell2013science}. Costs are applied on turnover events; slippage can be approximated via implementation shortfall vs. a reference mid price.

\subsection{Accounting-Quality Logs}
To ensure auditability, we persist:
\begin{itemize}
  \item Daily mark-to-market equity curve
  \item Realized and unrealized P\&L by ticker
  \item Trade blotter with fill reconciliation (broker fills matched back into records)
\end{itemize}
These records are required for credible research-to-production transfer.

% ==========================================================
\section{Results (DJI Research Validation)}
\subsection{Performance Summary}
The repository includes CSV outputs for equity curves, metrics, and OOS SHAP summaries. Figure \ref{fig:equity} and Figure \ref{fig:shap} are generated from the included result files.

\begin{figure}[H]
\centering
\includegraphics[width=0.92\textwidth]{equity_curve.png}
\caption{Cumulative equity curve (strategy vs benchmark). Use log scale in plotting for interpretability of compounding.}
\label{fig:equity}
\end{figure}

\begin{figure}[H]
\centering
\includegraphics[width=0.92\textwidth]{shap_summary.png}
\caption{Global SHAP importance (OOS). Macro / rate-regime variables increase in importance during stress periods.}
\label{fig:shap}
\end{figure}

\subsection{Interpretation}
A central claim is not ``a magic alpha,'' but that the explanation traces are \emph{economically coherent}: rate-related and volatility-related features rise during tightening or stress regimes, consistent with discount-rate intuition and risk repricing.

% ==========================================================
\section{Robustness Appendix: Monte Carlo via Block Bootstrap}
\label{sec:mc}
Backtests are single-path realizations. We complement point estimates with Monte Carlo resampling of the realized strategy return series using a \textbf{block bootstrap} to preserve short-horizon dependence \citep{politis1994stationary}. We report distributions of CAGR, volatility, and maximum drawdown.

\begin{figure}[H]
\centering
\includegraphics[width=0.92\textwidth]{mc_equity_fan.png}
\caption{Monte Carlo equity fan chart (block bootstrap). Median path with percentile bands.}
\label{fig:mc_fan}
\end{figure}

\begin{figure}[H]
\centering
\includegraphics[width=0.92\textwidth]{mc_summary_hist.png}
\caption{Monte Carlo distribution summary (CAGR, max drawdown).}
\label{fig:mc_hist}
\end{figure}

\noindent The code to generate these figures and LaTeX-ready tables is provided in the repository (see \texttt{code/monte\_carlo\_appendix.py}).

% ==========================================================
\section{Reproducibility: Minimal Code Excerpts}
Short excerpts illustrate the reproducibility approach (full code in repository).

\subsection{OOS SHAP Discipline (excerpt)}
\begin{lstlisting}[language=Python, caption={OOS-only SHAP computation (conceptual excerpt)}]
# for each walk-forward fold:
model.fit(X_train, y_train)
yhat_test = model.predict(X_test)

# SHAP computed on OOS only
explainer = shap.TreeExplainer(model)
shap_values = explainer.shap_values(X_test)
\end{lstlisting}

\subsection{Accounting Logs (excerpt)}
\begin{lstlisting}[language=Python, caption={Execution-grade logs: MTM equity and per-ticker P\&L (conceptual excerpt)}]
# end of day:
equity = cash + sum(position_qty[t] * last_price[t] for t in tickers)
write_equity_curve(date, equity)

# realized/unrealized by ticker:
unrealized[t] = position_qty[t] * (last_price[t] - avg_cost[t])
\end{lstlisting}

% ==========================================================
\section{Limitations and Future Work}
Key limitations include survivorship bias (if present), sensitivity to trading frictions, and the need for release-aware macro series. Future work will:
\begin{itemize}
  \item incorporate point-in-time universe membership,
  \item expand regime switching (macro stress filters, volatility targeting),
  \item validate on additional universes (ETFs, ATHEX).
\end{itemize}

% ==========================================================
\section{Data and Code Availability}
Code and results are available on GitHub and archived on Zenodo:
\begin{itemize}
  \item GitHub: \texttt{KarmirisP/quant-safe-xai-portfolio}
  \item Zenodo DOI: \texttt{10.5281/zenodo.18167108} \citep{zenodo_quantsafe_2026}
\end{itemize}

% ==========================================================
\section{Conflicts of Interest}
The author declares no conflicts of interest.

% ==========================================================
\section{AI-Assisted Writing Disclosure}
Portions of text were drafted and edited using AI-assisted tools. The author reviewed, verified, and takes full responsibility for the final content, code, and claims.

% ==========================================================
\bibliographystyle{plainnat}
% If you want to FORCE references to appear even if you forget a citation during drafting, uncomment:
% \nocite{*}
\bibliography{references}

\end{document}
